\documentclass{article}
\usepackage[numbers]{natbib}
\usepackage{blindtext}
\usepackage{titlesec}
\usepackage{amsmath}
\usepackage{hyperref}
\hypersetup{
    colorlinks=true,
    linkcolor=blue,
}
\bibliographystyle{unsrtnat}
\usepackage[utf8]{inputenc}

\title{Theory Group Seminar Notes}
\author{Rishit Dagli}
\date{October 2022}

\begin{document}

\maketitle

\tableofcontents

\section{Lower Bounds for Locally Decodable Codes from Semirandom CSP Refutation}

Combinatorial lower bounds for 3-query LDCs by \citet{https://doi.org/10.48550/arxiv.1911.10698}.

\subsection{Abstract}

A code $C$ is a q-locally decodable code (q-LDC) if one can recover any chosen bit $b_i$ of the $k$-bit message b with good confidence by randomly querying the $n$-bit encoding x on at most $q$ coordinates. Existing constructions of $2$-LDCs achieve blocklength $n = exp(O(k))$, and lower bounds show that this is in fact tight. However, when $q = 3$, far less is known: the best constructions have $n = subexp(k)$, while the best known lower bounds, that have stood for nearly two decades, only show a quadratic lower bound of $n \geq \Omega(k^2)$ on the blocklength.

In this talk, we will survey a new approach to prove lower bounds for LDCs using recent advances in refuting semirandom instances of constraint satisfaction problems. These new tools yield, in the $3$-query case, a near-cubic lower bound of $n \geq \tilde{\Omega}(k^3)$, improving on prior work by a polynomial factor in $k$.

\bibliography{references}

\end{document}
