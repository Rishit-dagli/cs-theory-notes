\documentclass{article}
\usepackage[numbers]{natbib}
\usepackage{blindtext}
\usepackage{titlesec}
\usepackage{amsmath}
\usepackage{hyperref}
\hypersetup{
    colorlinks=true,
    linkcolor=blue,
    urlcolor=blue,
}
\bibliographystyle{unsrtnat}
\usepackage[utf8]{inputenc}

\title{Theory Group Seminar Notes}
\author{Rishit Dagli}
\date{October 2022}

\begin{document}

\maketitle

\tableofcontents
\clearpage

\addcontentsline{toc}{section}{Introduction}
\section*{Introduction}

These are my notes for the seminars that happen in the \href{https://www.cs.toronto.edu/theory/}{Theory Group} at The University of Toronto. Many thanks to \href{http://www.cs.toronto.edu/~bor/}{Professor Allan Borodin} for allowing me to attend the Theory Group seminars and helping out. An online version of these notes are available at \url{https://rishit-dagli.github.io/cs-theory-notes}.\\

The Theory Group focuses on theory of computation. The group is interested in using mathematical techniques to understand the nature of computation and to design and analyze algorithms for important and fundamental problems.\\

The members of the theory group are all interested, in one way or another, in the limitations of computation: What problems are not feasible to solve on a computer? How can the infeasibility of a problem be used to rigorously construct secure cryptographic protocols? What problems cannot be solved faster using more machines? What are the limits to how fast a particular problem can be solved or how much space is needed to solve it? How do randomness, parallelism, the operations that are allowed, and the need for fault tolerance or security affect this?

\newpage

\section{Lower Bounds for Locally Decodable Codes from Semirandom CSP Refutation}

The related paper: Combinatorial lower bounds for 3-query LDCs by \citet{https://doi.org/10.48550/arxiv.1911.10698}. Seminar by \href{https://www.cs.cmu.edu/~pmanohar/}{Peter Manohar}.

\subsection{Abstract}

A code $C$ is a q-locally decodable code (q-LDC) if one can recover any chosen bit $b_i$ of the $k$-bit message b with good confidence by randomly querying the $n$-bit encoding x on at most $q$ coordinates. Existing constructions of $2$-LDCs achieve blocklength $n = exp(O(k))$, and lower bounds show that this is in fact tight. However, when $q = 3$, far less is known: the best constructions have $n = subexp(k)$, while the best known lower bounds, that have stood for nearly two decades, only show a quadratic lower bound of $n \geq \Omega(k^2)$ on the blocklength.\\

In this talk, we will survey a new approach to prove lower bounds for LDCs using recent advances in refuting semirandom instances of constraint satisfaction problems. These new tools yield, in the $3$-query case, a near-cubic lower bound of $n \geq \tilde{\Omega}(k^3)$, improving on prior work by a polynomial factor in $k$.

\newpage
\bibliography{references}

\end{document}
